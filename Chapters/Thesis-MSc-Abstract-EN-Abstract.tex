% #############################################################################
% Abstract Text
% !TEX root = ../main.tex
% #############################################################################
% reset acronyms
\acresetall
% use \noindent in firts paragraph
\noindent
% In this work the Method of Fundamental Solutions (MFS) is applied in the study of the spectrum of the Dirac Operator with infinite mass boundary conditions and in the study of transmissions problems encompassing the Poisson Equation. We start by presenting the theory surrounding both topics, present a detailed approach for the MFS, both in theoretical terms and numerical implementation
This thesis delves into the application of the Method of Fundamental Solutions (MFS), a meshless technique, to address a duo of distinct partial differential equation (\ac{PDE}) problems. Meshless methods provide an alternative to the standard mesh-based approaches, especially suited for intricate geometries. This study is centered on two focal points: firstly, the spectral analysis of the Dirac operator with infinite mass boundary conditions, investigated through large-scale simulations; secondly, the resolution of transmission problems involving the Poisson equation within polygonal and curved domains.

Within the MFS framework, the spectral behavior of the Dirac operator is systematically explored, both verifying existing conjectures and postulating fresh insights. The study also covers transmission problems with the Poisson equation, utilizing singularity subtraction techniques to refine accuracy of the method.

Comprising six chapters, this thesis establishes foundational theory, rigorously introduces and implements the MFS, incorporating strategies to address inherent limitations. The results presented underscore the method's validity in addressing intricate \ac{PDE} problems, thereby showcasing the effectiveness of meshless methods.