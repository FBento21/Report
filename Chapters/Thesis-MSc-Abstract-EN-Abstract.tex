% #############################################################################
% Abstract Text
% !TEX root = ../main.tex
% #############################################################################
% reset acronyms
\acresetall
\noindent
This dissertation studies the application of the Method of Fundamental Solutions (MFS), a meshless technique, to address a duo of Partial Differential Equations (PDEs) problems. Meshless methods provide an alternative to the standard mesh-based approaches, especially suited for intricate geometries. This study is centered on two focal points: firstly, the spectral analysis of the Dirac operator with infinite mass boundary conditions, investigated through large-scale simulations; secondly, the resolution of transmission problems involving the Poisson equation within polygonal and curved domains.

Within the MFS framework, the spectral behavior of the Dirac operator is explored, both verifying existing conjectures and postulating new ones. The study also covers transmission problems with the Poisson equation, utilizing singularity subtraction techniques to improve the accuracy of the method.

This thesis starts by establishing the necessary theory, rigorously introduces and implements the MFS, incorporating strategies to address inherent limitations. The results presented underscore the method's validity in addressing challenging PDE problems, showcasing the effectiveness of meshless methods.