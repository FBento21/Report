% #############################################################################
% Agradecimentos / Acknowledgments
% !TEX root = ../main.tex
% #############################################################################
À primeira vista, escrever a secção de Agradecimentos quase que toca a formalidade, pequenos parágrafos, uma minuta, curtas linhas sem floreados ou sem a escrita técnica que premeia o resto do trabalho. Porém, como no fim de qualquer etapa, tem o condão de nos levar a reavaliar todo o nosso percurso e toda a gente que tornou este trabalho possível.

Assim, o papel de destaque final vai para os meus orientadores, os professores Juha Videman e Pedro Antunes. Estou-lhes muito grato por aceitarem fazer este percurso comigo, pela disponibilidade que sempre demonstraram, pelas dúvidas esclarecidas, pela experiência e conhecimento que comigo partilharam; e claro, pela motivação e confiança que me deram e tiveram em mim.

Não poderei deixar ainda de agradecer ao Centro de Análise Matemática, Geometria e Sistemas Dinâmicos (CAMGSD), pela bolsa que me foi atribuída e sob a qual tive a oportunidade de efetuar este trabalho.

Claro, aos meus colegas e amigos que me acompanharam, pelas discussões e conversas, pelo apoio que me deram, este que foi sem dúvida importantíssimo em vários pontos, o meu obrigado. Vários houveram que ficaram pelo caminho, outros tantos que nunca souberam que dele fizeram parte, e uns poucos que de um estreita estrada fizeram áleas: a também esses agradeço (e muito!).

Mas quem nunca ficou pelo caminho foi a minha família, aos meus pais, à minha irmã, aos meus avós. Este trabalho não teria sido possível sem o vosso incondicional apoio e pelo que sempre me proporcionaram. O meu grande obrigado.

