% #############################################################################
% This is Chapter 6
% !TEX root = ../main.tex
% #############################################################################
% Change the Name of the Chapter i the following line
\fancychapter{Conclusion}
% \cleardoublepage
% The following line allows to ref this chapter
\label{chap:conclusion}

In this dissertation, the Method of Fundamental Solutions was used within the context of the Dirac operator with infinite mass boundary conditions, as well as in transmission problems involving the Poisson equation. This work also aimed to contribute to the comprehension of the method's capabilities and limitations through these applications while presenting its theoretical background.

Concerning the Dirac operator, the extension of the proof regarding the absence of separation of variables in polar coordinates is interesting in a numerical point of view: although anticipated, this extension complicates the utilization of specific solutions describing angular behavior near corner tips. Numerical simulations proved valuable in validating previous conjectures and generating new ones, reaffirming the \ac{MFS}'s efficacy in addressing specific challenges and solidifying its practical utility within the domain of numerical methods for Partial Differential Equations. Numerical evidence supporting the generalizations of the Faber-Krahn inequality and the Ashbaugh-Benguria theorem for both polygonal and smooth domains was established, further enhancing the credibility of these conjectures. Additionally, the unexpected behavior of the third eigenvalue in the Dirac operator's spectrum was revealed, signaling potential disparities from the behavior of the Laplace operator, particularly at the beginning of the spectrum.

The application of the Method of Fundamental Solutions with particular solutions to transmission problems is also significant, since it is the first study that introduces the use of particular solutions to better describe the solution's behavior near corners in these type of \ac{PDE}s. Although this technique was previously employed for other types of \ac{PDE}s, its application to transmission problems represents a novel advancement, expanding the practical utility of the method.

Nonetheless, numerous questions remain unanswered, setting the stage for future work. The study spectrum of the Dirac operator, alongside its distinct behavior compared to the Laplace operator, requires further investigation. This includes the behavior of the second eigenvalue and the optimality of shapes for various eigenvalues. Future explorations may uncover valuable insights that can might be useful for such theoretical problems, advancing our understanding of spectral geometry. Is the optimal shape for the second eigenvalue still two disjoint balls with the same volume, and does this insight provide a basis for understanding higher eigenvalues? Can the Method of Fundamental Solutions be adapted to handle topological changes, like connectedness, and how can the behavior of eigenvalues in such cases be predicted? Notice that for the third eigenvalue we still do not know if the optimal shape is even connected. If it is not, then the shape presented is not the optimal one. For the Laplacian, these topological questions can be addressed using Theorem \ref{wolf_keller}, but is there any analogous results for the Dirac operator with infinite mass boundary conditions?

Similarly, the transmission problem presents its own set of questions when using the Method of Fundamental Solutions. Is it possible to derive an \textit{a posteriori} error estimate, such that the error between an exact solution and a numerical solution is bounded in some norm? Is it possible to use different particular solutions whose behavior better adapt to the problem? How can these particular solutions be tailored to yield more accurate results on the interface between domains? Addressing these issues could further refine the Method of Fundamental Solutions and broaden its applicability. 

Our objective remains to continue this study, tackle these problems, and adapt these methods to push the boundaries of the already achieved results. This dissertation represents another step towards more accurate numerical methods for complex Partial Differential Equations, with the potential to uncover new insights into the behaviors of these mathematical entities, and to develop techniques that may aid in the solution of these questions in future research.