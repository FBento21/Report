% #############################################################################
% This is Chapter 6
% !TEX root = ../main.tex
% #############################################################################
\fancychapter{Conclusions}
% \cleardoublepage
\label{chap:conclusion}

In this dissertation, the Method of Fundamental Solutions (\ac{MFS}) was used within the context of the Dirac operator with infinite mass boundary conditions, as well as in transmission problems involving the diffusion equation. Besides presenting the theoretical foundations of the MFS, this work also aimed at contributing to the comprehension of the method's capabilities and limitations through the applications.

In regard to the Dirac operator, the extension of the proof on the absence of separation of variables in polar coordinates near a corner's tip is interesting from a numerical point of view. Although anticipated, this extension complicates the use of specific solutions near a corner. Numerical simulations proved valuable in validating previous conjectures and generating new ones, both reaffirming the \ac{MFS}'s efficacy in addressing specific challenges and solidifying its practical usefulness for the numerical approximation of Partial Differential Equations. Numerical evidence supporting the generalizations of the Faber-Krahn inequality and the Ashbaugh-Benguria Theorem, both in polygonal and in smooth domains was given, further enhancing the veracity of these conjectures. Additionally, an unexpected behavior of the third eigenvalue in the Dirac operator's spectrum was revealed, signaling potential disparities with respect to the behavior of the Laplace operator, even at the beginning of the spectrum. It is also worth noticing that such a finding only appears to be valid for some mass ranges, also indicating the spectrum's dependence on the mass \(m\).

The application of the Method of Fundamental Solutions with particular solutions to transmission problems is also significant since it is the first study that introduces the use of particular solutions to better describe the solution's behavior near corners in these types of \acp{PDE}. Although this technique was previously employed for other types of \acp{PDE}, its application to transmission problems represents a novel advancement, expanding the practical utility of the method.

Nonetheless, numerous questions remain unanswered, setting the stage for future work. The spectrum of the Dirac operator, alongside its distinct behavior compared to the Laplace operator, requires further investigation. This includes the behavior of the second eigenvalue and the shape of the optimal domain for various eigenvalues. Future explorations may uncover valuable insights that can be useful for such theoretical problems, advancing our understanding of spectral geometry. Is the optimal shape for the second eigenvalue still two disjoint balls with the same volume, and does this insight provide a basis for understanding higher eigenvalues? Can the Method of Fundamental Solutions be adapted to handle topological changes, like connectedness, and how can the behavior of eigenvalues in such cases be predicted? Notice that for the third eigenvalue, we still do not know if the optimal shape is even connected. If it is not, then the shape presented in this work is not the optimal one. For the Laplacian, these topological questions can be addressed using Theorem \ref{wolf_keller}, but is there any analogous result for the Dirac operator with infinite mass boundary conditions?

Similarly, the transmission problem presents its own set of questions and problems when using the Method of Fundamental Solutions. A result that is not addressed in this work concerns a potential \textit{a posteriori} error estimate. Such an estimate would aim to bound, both from above and from below, the error between an exact solution and a numerical solution by an error estimator depending only on the discrete solution and (possibly) the data. Is it possible to use different particular solutions whose behavior better adapts to the problem? How can these particular solutions be tailored to yield more accurate results on the interface between domains? Addressing these issues could further refine the Method of Fundamental Solutions and broaden its applicability. 

This dissertation represents a step towards finding more accurate numerical methods for complex Partial Differential Equations, with the potential to bring new insights into the investigation of the behavior of their solutions. Our objective remains to continue this study, tackle these problems, and adapt these methods to push the boundaries of the already achieved results. 