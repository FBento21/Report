% #############################################################################
% This is Chapter 1
% !TEX root = ../main.tex
% #############################################################################
\fancychapter{Introduction}
% \cleardoublepage
\label{chap:intro}

\section{On the applications of the Method of Fundamental Solutions}

\acp{PDE} serve as fundamental tools for modeling a wide spectrum of phenomena across scientific disciplines, ranging from engineering and physics to biology and finance. Given the complexity and infeasibility of deriving analytical solutions for many cases, accurate numerical solutions have become imperative. While well-established methods like finite differences and finite elements are available for a wide range of \acp{PDE}, meshless methods offer an effective alternative, particularly for intricate geometries. This dissertation studies the \ac{MFS}, a meshless technique, investigating its applications in solving two distinctive problem domains: the spectral analysis of the Dirac operator under infinite mass boundary conditions, and transmission problems involving the Poisson equation within polygonal and curved domains.

Emerging in the latter part of the previous century, meshless methods provide an alternative to traditional mesh-based approaches, circumventing the challenges of mesh generation in complex geometries. Drawing inspiration from potential and integral equations theory, the more recent Method of Fundamental Solutions approximates solutions by exploiting the fundamental solutions of governing \acp{PDE}. It has garnered attention for solving eigenvalue problems, as seen in \cite{alves2013method}, \cite{reutskiy2006method}, and \cite{antunes2011inverse}. We aim to uncover the \ac{MFS}'s capabilities in addressing various \acp{PDE} challenges, thereby providing valuable insights into the systems under examination.

The spectral analysis of the Dirac operator under infinite mass boundary conditions, a problem classified as pivotal in shape optimization theory by \cite{krejcirik_larson_lotoreichik_2019}, plays a critical role not only in our comprehension of quantum mechanics and quantum field theory but also in engineering applications, for example in the study of the so-called \textit{Dirac materials} like graphene. Our focus is centered on comprehending the spectral behavior of this operator. By employing the Method of Fundamental Solutions we strive to offer numerical insights that both validate existing conjectures and engender new ones.

Shifting our attention to transmission problems for the Poisson equation, which holds significance in fields like heat conduction, electromagnetism, and contact mechanics, we employ the \ac{MFS} to investigate solutions within polygonal and curved domains. This study goes into the complexities introduced by interfaces and compatibility conditions in such scenarios. In order to increase the precision of the method, we incorporate methodologies to enhance the \ac{MFS}, integrating singularity subtraction techniques to heighten accuracy, especially in proximity to domain's corners. Importantly, this study marks the first instance of utilizing this technique with the Method of Fundamental Solutions for these specific problems.

In subsequent sections, comprising theoretical foundations and numerical methodologies, our objective is to present a straightforward perspective of the \ac{MFS}'s role in addressing complex \acp{PDE} problems. This study not only furthers our understanding of meshless approaches but also bridges the gap between numerical simulation and theoretical research, serving as a source of new and challenging problems.



% \textcolor{violet}{This is an example of Tracking} \replaced[id=JO]{Changes}{Xanges} (in this case a replacement) by different authors in the document. The Text can additionally be modified by \added[id=PT]{adding} new text or by deleting \deleted[id=MN]{wrong} inadequate text. Author can manipulate changes \replaced[id=PT]{introduced by each author\deleted[id=MN]{, as adequate}}{intrroduced by other authors}.

 %#############################################################################

% \textcolor{violet}{You can use in-paragraph lists with this construct for: 
% \begin{inparaenum}[(a)]
% \item first case;
% \item second case; and
% \item third case,
% \end{inparaenum}
% making the text organized and fluid.}

% #############################################################################
\section{Thesis Overview}

This thesis is structured into six distinct chapters, each organized as follows:
\begin{itemize}
\item Chapter \ref{chap:Preliminaries} introduces foundational concepts in Functional Analysis and Partial Differential Equations. While the majority of these results can be found in classical references and are often covered in graduate courses, they serve as crucial underpinnings for the subsequent chapters. Notably, Chapter \ref{chap:numerical} draws heavily upon these concepts to establish the theoretical framework of the Method of Fundamental Solutions. 

\item Chapter \ref{chap:problem_introduction} serves as an introduction to the problems investigated within this thesis and is divided into three distinct sections. In the initial section, some analysis of the Laplace operator is done, presenting established results and conducting a small literature review. Although not directly connected with the study of the Dirac operator, the similarities between the two lead to the conjecture that significant findings of the Laplace operator could extend to the Dirac problem with infinite mass boundary conditions. This section also serves as a ground for the formulation of new conjectures concerning the Dirac operator's spectrum.

A subsequent portion of this chapter focuses on the exploration of the Dirac operator. It introduces the operator, elucidates some of its properties, and highlights its spectral characteristics. Additionally, a concise yet insightful proof demonstrates the absence of separable solutions in polar coordinates, extending what was previously known for cartesian coordinates. Recent conjectures postulated by field experts are presented, and novel conjectures, influenced by the prior analysis of the Laplace operator, are introduced. These conjectures subsequently become subjects of investigation using the \ac{MFS}, enabling a comprehensive exploration of their validity and implications.

Finally, the third section of this chapter centers on the Poisson transmission problem. It establishes the problem's context and its relationship with the Poisson equation when featuring a discontinuous source term. This section adopts a modern approach to analyze the relation between the transmission problem and the classical Poisson equation. This examination is important for the subsequent application of the \ac{MFS}, and it will be needed to theoretically justify the use of the method in Chapter \ref{chap:numerical}.

\item Chapter \ref{chap:numerical} introduces the Method of Fundamental Solutions, and presents the various density proofs which justify this numerical method for the various problems, improving both in the rigor and details. It also presents convergence and stability results, the advantages and disadvantages of the method, and different ways to address them, specifically an enrichment technique using particular (angular) solutions responsible for singularity subtraction. Finally, its numerical implementation and a direct search algorithm used to find the eigenvalues are presented.
\item In Chapter \ref{chap:implement}, we present our numerical findings, organized into two distinct sections. The initial section examines the Dirac operator with infinite mass boundary conditions, involving extensive large-scale simulations. Subsequently, outcomes for various domain shapes, including quadrilaterals, triangles, general \(n\)-side polygons for \(n=5,6,7,8\), and smooth domains, are outlined, emphasizing the pertinent discoveries. This section concludes by addressing an unconstrained minimization problem aimed at identifying optimal shapes. The second section is dedicated to the transmission problem, focusing on achieved numerical errors and the utilization of enrichment techniques to enhance the method's accuracy in such scenarios. These results are enhanced with visual aids and concise tables summarizing our findings.
\item Finally, Chapter \ref{chap:conclusion} finish this work, presents the relevant conclusions of this thesis, and proposes some future work related to this research topic.
\end{itemize}

% \begin{align*}
%     \begin{cases}
%         x^2\\
%     \end{cases}
% \end{align*}

% \begin{equation*}
%     y-x
% \end{equation*}