% #############################################################################
% RESUMO em Português
% !TEX root = ../main.tex
% #############################################################################
% use \noindent in firts paragraph
% reset acronyms
\acresetall
\noindent
Esta dissertação investiga a aplicação do Método das Soluções Fundamentais (\ac{MSF} em português), um método sem malha, adereçando dois distintos problemas em Equações de Derivadas Parciais (EDPs). Os métodos sem malha são uma alternativa aos clássicos métodos com malha e são particularmente adequados a geometrias mais complexas. Este estudo centra-se em dois pontos principais: primeiramente, na análise espetral do operador de Dirac com condições de fronteira de massa infinita, que foi investigado usando simulações de larga escala; em segundo lugar, na resolução de problemas de transmissão que envolvem a equação de Poisson, tanto em domínios poligonais e curvos.

Sob a estrutura do \ac{MSF}, o comportamento espetral do operador de Dirac é explorado sistematicamente, verificando conjeturas existentes e postulando novos resultados. Este estudo cobre também problemas de transmissão com a equação de Poisson, utilizando técnicas de subtração de singularidade que permitem melhorar a precisão do método.

Tendo seis capítulos, esta tese estabelece teoria basilar, introduz e implementa o \ac{MSF} rigorosamente, incorporando estratégias que abordam as suas inerentes limitações. Os resultados apresentados sublinham a validade do método em resolver problemas de EDPs complicados, mostrando assim a eficácia de métodos sem malha.